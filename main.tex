\documentclass[12pt,letterpaper]{article}

% Idioma y codificación
\usepackage[utf8]{inputenc}
\usepackage[spanish,es-tabla]{babel}
\usepackage{lmodern}

% Márgenes y formato
\usepackage[left=3cm,right=2.5cm,top=3cm,bottom=3cm]{geometry}
\usepackage{fancyhdr}
\pagestyle{fancy}
\fancyhf{}
\fancyhead[L]{Bitácora de Proyecto}
\fancyhead[R]{\thepage}

% Tablas y listas
\usepackage{longtable}
\usepackage{array}
\usepackage{multirow}
\usepackage{booktabs}
\usepackage{enumitem}

% Figuras y colores
\usepackage{graphicx}
\usepackage{xcolor}
\usepackage{caption}
\captionsetup[figure]{justification=centering}

% Hipervínculos
\usepackage[colorlinks=true, linkcolor=blue, urlcolor=blue, citecolor=black]{hyperref}

% Bibliografía
\usepackage[backend=bibtex,style=ieee,biblabel=dot]{biblatex}
\addbibresource{references.bib}

% Estilo de secciones
\usepackage{titlesec}
\titleformat*{\section}{\bfseries\large}
\titleformat*{\subsection}{\bfseries\normalsize}

%%%%%%%%%%%%%%%%%%%%%%%%%%%%%%%%%%%%%%%%%%%%
%                Documento                 %
%%%%%%%%%%%%%%%%%%%%%%%%%%%%%%%%%%%%%%%%%%%%
\begin{document}

% Portada
\begin{center}
    \textbf{\LARGE{Bitácora del Proyecto/Taller}} \\[6mm]
    \textbf{\large{Puerta con contraseña / Diseño de un decodificador}} \\[4mm]
    \textbf{\large{Fundamentos de Arquitectura de Computadores (CE 1107)}} \\[4mm]
    \textbf{\large{Instituto Tecnológico de Costa Rica}} \\[6mm]

    % Integrantes centrados
    \textbf{Integrantes:} \\[4mm]
    Emanuel Chavarría Hernández — 2022205841 \\[2mm]
    Fernando Fuchs Mora — 2020144908

    \textbf{\today}\\[8mm]
\end{center}


%%%%%%%%%%%%%%%%%%%%%%%%%%%%%%%%%%%%%%%%%%%%
%              Entradas de bitácora        %
%%%%%%%%%%%%%%%%%%%%%%%%%%%%%%%%%%%%%%%%%%%%

\section{Sesión 1 -- 29 de agosto de 2025}
\subsection*{Actividades realizadas}
\begin{itemize}
    \item Se creó el repositorio para guardar y actualizar la presente bitácora.
    \item Como grupo se acordó leer los requerimientos del proyecto para llegar con dudas al profesor en la próxima clase o por medios virtuales.
\end{itemize}

\subsection*{Resultados obtenidos}
Se comenzó con el desarrollo del proyecto en una etapa básica.

\subsection*{Próximos pasos}
\begin{itemize}
    \item Comenzar con el proceso de simulación.
    \item Definir la tecnología que se va a utilizar.
    \item Definir el programa de simulación antes de construir el circuito.
\end{itemize}

\newpage

\section{Sesión 2 -- 30 de agosto de 2025}
\subsection*{Actividades realizadas}
\begin{itemize}
    \item Se comenzó con la investigación general de cómo desarrollar el proyecto y cuáles son los objetivos de realización.
    \item Se aclararon dudas iniciales del grupo sobre el proyecto con el profesor.
    \item Se compraron dos sensores de choque (\textit{shock}) de manera preliminar para comprobar si sirven para los objetivos del proyecto.
\end{itemize}

\subsection*{Resultados obtenidos}
Se generó un plan de trabajo inicial.

\subsection*{Próximos pasos}
\begin{itemize}
    \item Se definió que el estudiante Emanuel Chavarría Hernández va a comenzar con el proceso inicial de realización de la simulación.
    \item Se puso en pausa el trabajo en el proyecto para atender otras responsabilidades académicas por parte de los integrantes.
\end{itemize}

\section{Sesión 3 -- 4 de septiembre de 2025}
\subsection*{Actividades realizadas}
\begin{itemize}
    \item Se definió el uso del programa Logisim-Evolution para la simulación.
    \item Surgieron nuevas dudas, esperando ser resueltas por parte del profesor.
    \item Se definió un circuito serializador, a la espera de ser aprobado por el profesor.
\end{itemize}

\subsection*{Resultados obtenidos}
Se contó con el circuito serializador de manera preliminar, terminado.

\subsection*{Próximos pasos}
\begin{itemize}
    \item Resolver las dudas con el profesor lo más pronto posible.
    \item Intentar definir qué circuitos integrados comprar.
    \item Trabajar simultáneamente tanto en el taller como en el proyecto.
\end{itemize}

\section{Sesión 4 -- 8 de septiembre de 2025}
\subsection*{Actividades realizadas}
\begin{itemize}
    \item Se resolvieron las dudas con el profesor.
    \item Se comenzó con la construcción del circuito combinatorio para reconocer el patrón por medio de simulación.
    \item Se definió el BCD a utilizar en el proyecto/taller.
    \item Se planteó que, en el display de 7 segmentos, se muestre un 1 o un 2 dependiendo de si se abre o se cierra.
\end{itemize}

\subsection*{Resultados obtenidos}
Se avanzó de forma significativa en el proyecto/taller.

\subsection*{Próximos pasos}
\begin{itemize}
    \item Confirmar el uso del 7 segmentos (representando con 1 y 2).
    \item Definir materiales por comprar para montar el circuito de manera física.
\end{itemize}

\section{Sesión 5 -- 9 de septiembre de 2025}
\subsection*{Actividades realizadas}
\begin{itemize}
    \item Se terminó el circuito combinacional que detecta los patrones de entrada y salida de manera simulada.
    \item Se resolvieron problemas con la representación en el 7 segmentos, logrando usar $A$ para abrir y $C$ para cerrar.
    \item Se definió el uso de la tecnología TTL para el resto del proyecto.
    \item Se añadieron los archivos de simulación al repositorio dedicado al proyecto, para un mejor manejo de versiones.
\end{itemize}

\subsection*{Resultados obtenidos}
Se finalizó la simulación del taller.

\subsection*{Próximos pasos}
\begin{itemize}
    \item El estudiante Fernando Fuchs trabajará en montar el circuito en la herramienta Tinkercad para mayor facilidad a la hora de construirlo en físico.
    \item Se definió comprar los materiales entre el 10 y el 11 de septiembre.
    \item El estudiante Fernando Fuchs trabajará en las tareas restantes (investigar sobre el motor y accionador de puerta, etc.).
\end{itemize}

\end{document}
