\documentclass[12pt,letterpaper]{article}

% Idioma y codificación
\usepackage[utf8]{inputenc}
\usepackage[spanish,es-tabla]{babel}
\usepackage{lmodern}

% Márgenes y formato
\usepackage[left=3cm,right=2.5cm,top=3cm,bottom=3cm]{geometry}
\usepackage{fancyhdr}
\pagestyle{fancy}
\fancyhf{}
\fancyhead[L]{Bitácora de Proyecto}
\fancyhead[R]{\thepage}

% Tablas y listas
\usepackage{longtable}
\usepackage{array}
\usepackage{multirow}
\usepackage{booktabs}
\usepackage{enumitem}

% Figuras y colores
\usepackage{graphicx}
\usepackage{xcolor}
\usepackage{caption}
\captionsetup[figure]{justification=centering}

% Hipervínculos
\usepackage[colorlinks=true, linkcolor=blue, urlcolor=blue, citecolor=black]{hyperref}

% Bibliografía
\usepackage[backend=bibtex,style=ieee,biblabel=dot]{biblatex}
\addbibresource{references.bib}

% Estilo de secciones
\usepackage{titlesec}
\titleformat*{\section}{\bfseries\large}
\titleformat*{\subsection}{\bfseries\normalsize}

%%%%%%%%%%%%%%%%%%%%%%%%%%%%%%%%%%%%%%%%%%%%
%                Documento                 %
%%%%%%%%%%%%%%%%%%%%%%%%%%%%%%%%%%%%%%%%%%%%
\begin{document}

% Portada
\begin{center}
    \textbf{\LARGE{Bitácora del Proyecto/Taller}} \\[6mm]
    \textbf{\large{Puerta con contraseña / Diseño de un decodificador}} \\[4mm]
    \textbf{\large{Fundamentos de Arquitectura de Computadores (CE 1107)}} \\[4mm]
    \textbf{\large{Instituto Tecnológico de Costa Rica}} \\[6mm]
    \today
\end{center}

\vspace{1cm}

\section*{Integrantes}
\begin{itemize}[label=\textbullet]
    \item Emanuel Chavarría Hernández 2022205841
    \item Fernando Fuchs Mora 2020144908
\end{itemize}


%%%%%%%%%%%%%%%%%%%%%%%%%%%%%%%%%%%%%%%%%%%%
%              Entradas de bitácora        %
%%%%%%%%%%%%%%%%%%%%%%%%%%%%%%%%%%%%%%%%%%%%

\section{Sesión 1 -- 29/agosto/2025}
\subsection*{Actividades realizadas}
\begin{itemize}
    \item Se crea el repositorio para guardar y actualizar la presente bitacora
    \item Como grupo se concuerda en leer los requerimientos del proyecto para llegar con dudas al profesor la próxima clase o por medio virtuales
\end{itemize}

\subsection*{Resultados obtenidos}
Se comienza con el desarrollo del proyecto en una etapa básica.


\subsection*{Próximos pasos}
\begin{itemize}
    \item Comenzar con el proceso de simulación.
    \item Definir la tecnologia que se va a utilizar.
    \item Definir el programa de simulación antes de construir el circuito.
\end{itemize}

\newpage

\section{Sesión 2 --  30/agosto/2025}
\subsection*{Actividades realizadas}
\begin{itemize}
    \item Se comienza con la investigación general de como desarrollar el proyecto y cuales son los objetivos de realización.
    \item Se aclaran dudas iniciales del grupo sobre el proyecto con el profesor.
    \item Se compran dos sensores tipo Schock de manera pre-liminar para comprobar si sirven para los objetivos del proyecto.
\end{itemize}

\subsection*{Resultados obtenidos}

\subsection*{Próximos pasos}
\begin{itemize}
    \item Se define que el estudiante Emanuel Chavarría Hernández va a comenzar con el proceso inicial de realización de la simulación.
    \item Se pone en pausa el trabajo en el proyecto para salir de otras responsabilidades académicas por parte de los integrantes.
\end{itemize}


\section{Sesión 3 --  4/setiembre/2025}
\subsection*{Actividades realizadas}
\begin{itemize}
    \item Se define el uso del programa Logisim-Evolution para la simulación
    \item Surgen nuevas dudas por parte, esperando ser resueltas por parte del profesor.
	\item Se define un cicruito serializador esperando a ser aprobado por el profesor.
\end{itemize}

\subsection*{Resultados obtenidos}

\subsection*{Próximos pasos}
\begin{itemize}
    \item Se busca resolver las dudas con el profesor lo mas antes posible
	\item Se intenta definir que circuitos integrados comprar
	\item Se trabaja simultaneamente tanto en el taller como el proyecto.
\end{itemize}



\printbibliography

\end{document}
